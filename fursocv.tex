

\documentclass[11pt,letterpaper]{report}

\usepackage[T1]{fontenc} % output T1 font encoding (8-bit) for accented characters as single glyph
\usepackage[strict,autostyle]{csquotes} % smart and nestable quote marks
\usepackage[USenglish]{babel} % regionalize hyphens, quote marks, etc automatically
\usepackage{microtype}% improve text appearance with kerning, etc
\usepackage{datetime} % enable formatting of date output
\usepackage{tabto}    % make nice tabbing
\usepackage{hyperref} % enable hyperlinks and pdf metadata
\usepackage{geometry} % manually set page margins
\usepackage{enumitem} % enumerate with [resume] option
\usepackage{titlesec} % allow custom section fonts
\usepackage{setspace} % custom line spacing

% what is your name?
\newcommand{\myname}{Francesco Urso}

% select default typefaces
\usepackage{lmodern}
% how far to tab for list items with left-aligned date: different fonts need different widths
\newcommand{\listtabwidth}{1.7cm}

% define font to use as document's title
\newcommand{\namefont}[1]{{\normalfont\bfseries\Huge{#1}}}

% set section heading fonts and before/after spacing
\SetTracking{encoding=*, family=\sfdefault}{30} % increase sans serif headings tracking
\titleformat{\section}{\lsstyle\sffamily\small\bfseries\uppercase}{}{}{}{}
\titlespacing{\section}{0pt}{30pt plus 4pt minus 4pt}{8pt plus 2pt minus 2pt}

% set subsection heading fonts and before/after spacing
\titleformat{\subsection}{\lsstyle\sffamily\footnotesize\bfseries}{}{}{}{}
\titlespacing{\subsection}{0pt}{16pt plus 4pt minus 4pt}{4pt plus 2pt minus 2pt}

% set page margins (assumes letter paper)
\geometry{body={6.5in, 9.0in},
    left=1.0in,
    top=1.0in}

% prevent paragraph indentation
\setlength\parindent{0em}

% set line spacing
\setstretch{0.9}

% define space between list items
\newcommand{\listitemspace}{0.25em}

% make unordered lists without bullets and use compact spacing
\renewenvironment{itemize}
{\begin{list}{}{\setlength{\leftmargin}{0em}
                \setlength{\parskip}{0em}
                \setlength{\itemsep}{\listitemspace}
                \setlength{\parsep}{\listitemspace}}}
{\end{list}}

% make tabbed lists so content is left-aligned next to years
\TabPositions{\listtabwidth}
\newlist{tablist}{description}{3}
\setlist[tablist]{leftmargin=\listtabwidth,
    labelindent=0em,
    topsep=0em,
    partopsep=0em,
    itemsep=\listitemspace,
    parsep=\listitemspace,
    font=\normalfont}

% print only the month and year when using \today
\newdateformat{monthyeardate}{\monthname[\THEMONTH] \THEYEAR}

% define hyperlink appearance and metadata for pdf properties
\hypersetup{
    colorlinks  = true,
    urlcolor    = black,
    citecolor   = black,
    linkcolor   = black,
    pdfauthor   = {\myname},
    pdfkeywords = {city planning, housing, street networks, transportation, urban design, urban informatics},
    pdftitle    = {\myname: Curriculum Vitae},
    pdfsubject  = {Curriculum Vitae},
    pdfpagemode = UseNone
}

\begin{document}
    \raggedright{}

    % display your name as the document title
    \namefont{\myname}

    % affiliation and contact info blocks
    \vspace{1em}
    \begin{minipage}[]{0.7\textwidth}
        % current primary affiliation, left-aligned
        Department of Physics \\
        University of Pisa
    \end{minipage}
    \begin{minipage}[]{0.1\textwidth}
        % contact info details, right-aligned
        \flushright{}
        \href{mailto:f.urso3@studenti.unipi.it}{f.urso3@unipi} \\
     
        \href{https://github.com/francescouni0}{github/francescouni0}
    \end{minipage}


    \section*{Education}

    \begin{tablist}

        %\item[Ph.D.] \tab{}City and Regional Planning, University of California, Berkeley, 2017
        \item[M.S.]  \tab{}Medical Physics, University of Pisa, 2022-
        \item[B.S.]  \tab{}Physics, University of Pisa, 2022

    \end{tablist}



    \section*{Research Experience}

    
    \subsection*{INFN, Pisa:} Performed measurements and 
    data analysis in the context of my master thesis. My work focused
    both on simulations using Geant4 and on experimental work conducted
    at the Medical Physics INFN laboratories in Pisa. I have been 
    involved in the calibration and use of state of the art 
    detectors(UTOF-PET) and in the development of a new detector for
    the measurement of prompt gamma rays in very high energy electron FLASH
    radiotherapy. I developed a deep learning pipeline for dose estimation 
    from prompt gamma distribution.
     \\Under the supervision of Prof. Maria Giuseppina Bisogni.
    \subsection*{Santa Chiara University Hospital and CISUP, Pisa:}
    I was able to setup and perform measurements at state of the art facilities
    for the treatment of cancer patients. I have been involved in data 
    acquisition and analysis at the TrueBeam linac accelerator and at the
    FLASH radiotherapy facility.\\ Under the supervision of Dr. Stefania Linsalata.
    \subsection*{CNR-INO, Pisa:} I have been involved in the calibration 
    and measurement setup of a high-gradient laser wakefield accelerator at the Intense Laser Irradiation Laboratory (ILIL).
    \\Under the supervision of Prof. Leonida A. Gizzi.
    \subsection*{Beam Test Facility(LNF), Frascati(RM) :}Took active part
    in the experimental setup and data acquisition in the context of the MORSE(Monitor for Flash Therapy)
    project. I took part in measurements of energy deposition in a plastic scintillator
    using a CCD camera. I have also gained experience in the operation and monitoring
    of a high energy beamline like the one at the Beam Test Facility.
    \\Under the supervision of Dr. Luca Foggetta.

    \section*{Conferences and Workshops}
    \begin{tablist}
        \item[2024] \tab{}Fondazione Bruno Kessler, Trento. Update meeting on the PRIN 
        project MORSE(Monitor for Flash Therapy). Presented the work on prompt gamma 
        monitoring in VHEE radiotherapy.

    \end{tablist}


    \section*{Projects}
    \begin{itemize}
        \item -Python and Matlab package containing different tools for feature exctraction, machine learning and deep learning analysis of Diffusion Tensor Imaging for Alzheimer's desease classification. 
         \\Available on github.
        \item -Geant4 simulation and Python image reconstruction of compton camera based on a plastic or silicon 
        scatterer and LYSO absorber. \\Available on github.    
    \end{itemize}

    \section*{Skills}
    \begin{itemize}
        \item -Programming Languages: Python, Matlab, C++
        \item -Machine Learning packages: Tensorflow, Keras, Pytorch, 
        Scikit-learn, OpenCV, Dicom, Nibabel
        
        
        \item -Data Analysis packages: Pandas, Numpy, Scipy, Matplotlib, 
        Seaborn
        \item -Software for Medical Physics: Raystation
        
        \item -Software for Particle Physics: Geant4, FLUKA, ROOT, GATE 
    \end{itemize}

    \section*{Languages}
    \begin{itemize}
        \item -Italian: Native
        \item -English: Fluent (C1)
        \item -French: Basic (A2)
    \end{itemize}


    %\section*{Publications}
%
    %\subsection*{Journal Articles}
%
    %\begin{tablist}
%
    %    \item[2021] \tab{}Urso, F., \& O'Sullivan, D. (2021). The impact of ride-hailing on urban form: Evidence from U.S. cities. \textit{Journal of the American Planning Association}, 87(4), 1--15.
%
    %    \item[2020] \tab{}Urso, F., \& O'Sullivan, D. (2020). The geography of Airbnb: A decade of global expansion. \textit{Urban Studies}, 57(9), 1--18.
    %    
    %\end{tablist}


    % display today's date as Month Year after a vertical space below the end of the text
    \begin{center}
        \vfill
        Updated \monthyeardate\today
    \end{center}

\end{document}
